\documentclass{amsart}

\usepackage{times}
\usepackage[T1]{fontenc}
\usepackage[latin1]{inputenc}
\usepackage{graphicx}
\usepackage{amssymb}
\usepackage{comment}

% \parindent0pt\parskip3pt
\parskip2ptplus0.5ptminus0.3pt

\def\C{\mathbb{C}}
\def\R{\mathbb{R}}
\def\Q{\mathbb{Q}}
\def\Z{\mathbb{Z}}
\def\ds{\displaystyle}
\def\hat{\widehat}
\def\O{\mathcal{O}}

\newcommand{\Img}[2]{\includegraphics[width=#1truecm]{#2}}

\newtheorem{theorem}{Theorem}
\newtheorem{lemma}{Lemma}
\newtheorem{corollary}{Corollary}

\begin{document}

\date{\today}
\title{On Differences of Zeta Values}
\author{Philippe Flajolet and Linas Vepstas}

\begin{abstract}
Finite differences of the Riemann zeta function at the integers
are explored. Such quantities, which occur as coefficients in Newton series representations, 
have surfaced in works of Ma{\'s}lanka, Coffey, B{\'a}ez-Duarte, Voros and others.
We apply the theory of N\"orlund-Rice integrals in conjunction with
the saddle point method  and derive precise 
asymptotic estimates. The method extends to Dirichlet $L$-functions,
in which case our estimates appear to be partly related to
earlier investigations surrounding Li's criterion for the Riemann hypothesis.
\end{abstract}

\maketitle

\section*{Introduction}


 


In recent  times, a variety of  authors have, for a variety  of reasons,
been led  to considering   properties  of  representations of   the
Riemann zeta function $\zeta(s)=\sum 1/n^s$  as a \emph{Newton interpolation
series}.  Amongst the many  possible  forms, we single out
the one relative to a regularized version of Riemann zeta, namely,
\begin{equation}\label{eq1}
\zeta(s)-\frac{1}{s-1}=\sum_{n=0}^\infty
(-1)^n b_n \binom{s}{n},
\end{equation}
where $\binom{s}{n}$ is a binomial coefficient:
\[
\binom{s}{n}:=\frac{s(s-1)\cdots(s-n+1)}{n!}.
\]
Corollary~\ref{newton-cor} in Section~\ref{conv-sec}
establishes that the representation~\eqref{eq1} is
valid throughout the complex plane,
its coefficients being determined by a general formula in the calculus of finite
differences~\cite{Jordan65,Milne81,Norlund54}:
\begin{equation}\label{eq2}
b_n=
% \sum_{k=0}^n
% \binom{n}{k}(-1)^k\left[\zeta(k)-\frac{1}{k-1}\right]
n(1-\gamma-H_{n-1})-\frac12+\sum_{k=2}^n \binom{n}{k}
(-1)^k\zeta(k),
\end{equation}
(Here,
$H_n=1+\frac12+\cdots\frac1n$ are the harmonic numbers.)
Although the terms in the sum defining~$b_n$ become
exponentially large, the values of the $b_n$ are exponentially small,
% \[
% b_n=\mathcal{O}\left(n^{1/4}e^{-2\sqrt{\pi n}}\right),
% \]
while exhibiting a curious oscillatory behavior. 
We shall indeed prove the estimate (Theorem~\ref{zetacoeff-thm} of Section~\ref{sadzeta-sec}) 
% that, to the next leading order,
\begin{equation}\label{asympform}
b_n =\left(\frac{2n}{\pi}\right)^{1/4}e^{-2\sqrt{\pi n}}\cos\left(2\sqrt{\pi n}+\frac{3\pi}{8}\right)
+\mathcal{O}\left(n^{-1/4} e^{-2\sqrt{\pi n}}\right).
\end{equation}

% \footnote{
% LV: Based on my calculations, the amplitude of this expression is too 
% large by a factor of $\sqrt{2}$, and the phase is off by $\pi$ (that is, 
% the expression should be negative).  This is supported by my numerical 
% calculations as well. The factor of two may or may not be due to a factor
% of two that seems to be missing in \eqref{om012} (see the footnote there).
% }

A prime reason for  interest in the representation~\eqref{eq1} and the
companion           coefficients~\eqref{eq2}         is     \emph{Li's
criterion}~\cite{Li97} for the   Riemann Hypothesis (RH).   Let $\rho$
range over the  nontrivial zeros of~$\zeta(s)$.   Li's theorem asserts
that RH is true if and only if all members of the sequence
\[
\lambda_n=\sum_\rho \left[1-\left(1-\frac{1}{\rho}\right)\right], \qquad n\ge0,
\]
are nonnegative. (Bombieri and Lagarias
 offer an insightful discussion of Li's criterion in~\cite{BoLa99}.) 
Coffey~\cite{Coffey05} has shown that the $\lambda_n$
can be alternatively  expressed as a  sum of  two terms, one  of which
(the easier one, though) is an elementary variant $\hat b_n$ of $b_n$.
Theorem~2 of~\cite{Coffey05}  amounts to the property that the coefficients $\hat b_n$ 
decrease to~0. As we shall see in Secion~\ref{dir-sec},
the methods originally developed for estimating~$b_n$
yield precise asymptotic information on~$\hat b_n$ as well.

The zeta function has also received attention in physics,
for its role in regularization and renormalization in 
quantum field theory. 
Motivated by such connections, 
Ma\'slanka introduced in~\cite{Maslanka01} 
what amounts to a Newton series representation of $(1-2s)\zeta(2s)$.
Further numerical observations relative to the corresponding coefficients
are presented in~\cite{Maslanka04}, which
have been subsequently vindicated by B\'aez-Duarte in~\cite{Baez03}.
In particular, B\'aez-Duarte's estimates imply that the coefficients
in the Newton series of a regularized version of~$(1-2s)\zeta(2s)$ decrease to~0
faster than any power of $1/n$.

A third independent reason was an attempt by one of us (Linas) around 2003 to obtain
alternative tractable expressions for the Gauss-Kuzmin-Wirsing operator of 
continued fraction theory [unpublished; see
\verb|http://linas.org/math/poch-zeta.pdf|]

In this essay,  we  approach the problem of  asymptotically estimating
differences of  zeta  values by means of   a combination  of  two well
established   techniques.   We   start   from   a   contour   integral
representation of these  differences as defined by~\eqref{eq2}  (for this
technique, see   especially N\"orlund's treatise~\cite{Norlund54}  and
the study~\cite{FlSe95}),  then proceed to estimate the corresponding
complex  integral by  means of the classical  saddle point method of asymptotic
analysis~\cite{deBruijn81,Olver74}. Our approach parallels a recent  preprint
of Voros~\cite{Voros05} (motivated by Li's criterion), 
which our results supplement by providing a fairly 
detailed asymptotic analysis of differences of zeta values.

The next section (\S\ref{newtser-sec}) will review the Newton series for the zeta function,
and give some generating functions for its coefficients. This is followed
by a brief review of some numerical results in \S\ref{exper-sec}. Section~\ref{norl-sec}
gives the N\"orlund integral representation for the series,
while the Section~\ref{sadzeta-sec} provides a careful saddle-point analysis 
of the resulting integrand.  Section~\ref{dir-sec} develops the corresponding analysis
for the Dirichlet $L$-functions.
 We end with a conclusion outlining other applications of N\"orlund integrals
in the realm of finite differences and zeta functions.




\section{Newton series and zeta values}\label{newtser-sec}


This section defines the Newton series for the Riemann zeta
that is to be studied, demonstrates some of its simple properties,
and gives some generating functions for its coefficients.
In this paper, a Newton series will be taken to be defined as

\begin{equation}\label{gennew}
\Phi(s)=\sum_{n=0}^\infty (-1)^n c_n \binom{s}{n}.
\end{equation}
Given a function $\phi(s)$, one may attempt to represent it 
in some region of the complex plane by means of 
such a series. Since the series $\Phi(s)$ terminates at $s=0,1,2,\ldots$,
the conditions $\phi(m)=\Phi(m)$ at the nonnegative integers
imply that the candidate sequence $\{c_n\}$ is linearly related to the
sequence of values $\{\phi(m)\}$ by
\[
\phi(m)=\sum_{n=0}^m (-1)^n c_n \binom{m}{n}.
\]
The triangular system can then be inverted to 
give (by the binomial transform~\cite{GrKnPa89} or by direct elimination)
\begin{equation}\label{gencoeff}
c_n = \sum_{k=0}^{n} \binom{n}{k} (-1)^k \phi(k), \qquad n=0,1,2,\ldots\,.
\end{equation}
This choice of coefficients for~\eqref{gennew}
determines the Newton series \emph{associated} to $\phi$.
The coincidence of 
the function $\phi$ and its associated series
$\Phi$ is, by construction, granted at least
at all the nonnegative integers. 
The validity
of $\Phi(s)=\phi(s)$  is often found to extend to large
parts of the complex plane, but this fact requires specific
properties much beyond the mere convergence of the series in~\eqref{gennew}.


% The equivalence
% of $\Phi(m)=\phi(m)$ can be extended to a region
% of the complex plane containing the integers, provided that the
% growth of the difference $\Phi(s)-\phi(s)$ is bounded in the imaginary
% direction. This result is known as \emph{Carlson's theorem}, and is
% presented in a more precise form later in this paper.

In the case of the Newton series for $\zeta(s)-1/(s-1)$, the general
relation~\eqref{gencoeff} provides the coefficients in the form
\begin{equation}\label{bn0}
b_n = s_0-ns_1+\sum_{k=2}^n \binom{n}{k} (-1)^k \left[\zeta(k)-\frac{1}{k-1}
\right],
\end{equation}
where
\begin{equation}\label{bn1}
s_0=\left[\zeta(s)-\frac{1}{s-1}\right]_{s=0}=\frac12,
\qquad
s_1=\lim_{s\to 1} \left[\zeta(s)-\frac{1}{s-1}\right] = \gamma.
\end{equation}
The harmonic numbers appear as
\begin{equation}\label{bn2}
\sum_{k=2}^n \binom{n}{k}\frac{(-1)^k}{k-1}=1-n+nH_{n-1}.
\end{equation}
Equations \eqref{bn0}, \eqref{bn1},   \eqref{bn2} 
then entail that   the
$b_n$, as  defined by~\eqref{eq2}, are indeed the  coefficients  of the Newton series
associated to~$\zeta(s)-1/(s-1)$. A proof of that the equality
$\Phi(s)=\zeta(s)-1/(s-1)$ holds for all complex~$s$ is given in Section~\ref{conv-sec},
following the asymptotic analysis of the coefficients~$b_n$ and based on
a Theorem of Carlson.

\smallskip

Before engaging in a detailed study of the~$b_n$, we note a few 
simple facts regarding their elementary properties. Set
\begin{equation}\label{deldef}
\delta_n:=\sum_{k=2}^n \binom{n}{k} (-1)^k \zeta(k),
\end{equation}
which are, up to minor adjustments, differences of zeta values at the
integers.  Defining the forward difference $\Delta f(x):=f(x+1)-f(x)$,
one has
\[
\delta_n= (-1)^n \Delta^n Z(x) \bigg|_{x=0},\]
where $Z(k)=\zeta(k)$ for $k\ge2$ and $Z(0)=Z(1)=0$.
Expanding the zeta function according to its definition and 
exchanging the order of summations in the resulting double sum shows that
\begin{equation}\label{simpdel}
\delta_n = \sum_{\ell\ge1} \left[\left(1-\frac{1}{\ell}\right)^n-1+\frac{n}{\ell}\right].
\end{equation}
This rather simple sum shows a remarkably complex behavior;
elucidating its behavior is one of the principal topics of this paper.


The ordinary generating function for the sequence $\{\delta_n\}$ 
is also of interest.
Given the classical expansion~\cite{WhWa27} 
of the logarithmic derivative $\psi(z)$ of the Gamma function,
\[
\psi(1+z)+\gamma=\zeta(2)z-\zeta(3)z^2+\zeta(4)z^3-\cdots,
\]
one finds
\begin{equation}\label{delogf}
\sum_{n\ge2}\delta_n z^n =\frac{z}{(1-z)^2}\left[\psi\left(\frac{1}{1-z}\right)
+\gamma\right].
\end{equation}
The exponential generating function for the sequence  $\{\delta_n\}$ reflects~\eqref{simpdel} and is even simpler:
% can be directly obtained from~\eqref{simpdel}: 
% multiply by~$z^n/n!$ and sum, to get
\begin{equation}\label{delegf}
\sum_{n\ge2}\delta_n \frac{z^n}{n!}
=e^z\sum_{n\ge2} \zeta(n) \frac{(-z)^n}{n!}=e^z \sum_{\ell\ge1}
\left[e^{-z/\ell}-1+\frac{z}{\ell}\right]
.
\end{equation}
% where the first form results directly from~\eqref{deldef},  the second from~\eqref{simpdel}.


\section{Experimental analysis}\label{exper-sec}

Detailed experiments on the $b_n$ coefficients conducted by one of us are
at the origin of the present paper. As it is usual when dealing with finite differences,
the alternating binomial sums giving the $b_n$ involve exponential
cancellation since the binomial coefficients get almost as large as $2^n$. 
We conducted evaluations of the $b_n$ up to $n\approx 5000$,
which requires computing zeta values up to several thousand digits of precision.
(Note that the zeta values can be computed rapidly to extremely high
precision using several efficient algorithms, which are available in several
symbolic computation packages and numerical libraries.)

A quick inspection of numerical data immediately reveals two features
of the constants~$b_n$: they are oscillatory with slowly increasing
period and their
absolute values are very rapidly decreasing. For instance\footnote{%
	The notation $x\doteq y$ designates a numerical approximation of
	$x$ by $y$ to the last decimal digit stated. 
}:
\[\renewcommand{\arraycolsep}{2truept}
\begin{array}{lllllllll}
b_1&\doteq&-	0.07721,\quad &
b_2&\doteq&-	0.00949,\quad &
b_5&\doteq& 	0.00071, \\
b_{10}&\doteq&-	0.00002, \quad &
b_{20}&\doteq& 	2.15965\cdot 10^{-9},\quad &
b_{50}&\doteq&-	1.08802\cdot 10^{-11}.
\end{array}\]

A numeric fit of the oscillatory behavior of the function may be made.
There are sign changes in the sequence $\{b_n\}$ at
\[
n=3,7,13,21,29,40,52,65,80,97,115,135,157,180,\ldots\,,
\]
the values growing roughly quadratically. A good fit for the $k$th zero is
provided by
\[
q(k)= \frac{\pi}{4}k^2+\frac{9\pi}{16}k+1,\]
rounded at the  nearest integer. (The precise values of the 
first two constants were inferred from
values of $n$ well in the range of several thousands.)
The quadratic polynomial is easily 
inverted to give the approximate oscillatory behavior of the $b_n$:
\[
s(n) = \sin\pi\left(2\sqrt{\frac{n}{\pi}}-\frac{9}{8}\right).\]

\begin{figure}

\begin{center}
\hbox{\Img{6}{exper1.jpg}\Img{6}{exper2.jpg}}
\end{center}

\caption{\label{exper-fig}
Numerical experiments with~$b_n$. Left: a plot of~$\log|b_n/s(n)|$.
Right: a plot of $b_n/\beta(n)$.}
\end{figure}
Once the oscillatory behavior has been disposed of, the task of quantifying
the general trend in the decrease becomes easier. A plot of the values 
of $\log|b_n/s(n)|$ displayed in Figure~\ref{exper-fig} (left)
has a parabolic aspect, which suggests that
\[
b_n \approx s(n) e^{-K\sqrt{n}} \qquad \hbox{with}\quad K\approx 3.6,\]
but a more precise fit  is  difficult. 

In summary, this together with similar experiments led us to conjecture 
\begin{equation}\label{conj}
\beta(n):=  \sin\pi\left(2\sqrt{\frac{n}{\pi}}-\frac{9}{8}\right)  e^{-K\sqrt{n}},
\qquad K=3.6\pm 0.1.
\end{equation}
as  a   rough approximation  to~$b_n$.  Figure~\ref{exper-fig} (right)
displays the ratios $b_n/\beta(n)$  for $n=10\,.\,.\,200$.   The trend  is
compatible furthermore with the presence  of
an   extra   factor   of     the  form   $-n^\kappa$   for   some
$\kappa\in(0,1)$. (The spikes
correspond  to    occasional inaccuracies  of  our sign-change function,    $s(n)$.) 
As we shall see, these empirical observations match reality quite well.


\section{The N\"orlund integral representation}\label{norl-sec}


Our approach to the asymptotic estimation of the $b_n$ relies on
a complex integral representation of finite differences of an analytic function,
to be found in N\"orlund's classic treatise~\cite[\S VIII.5]{Norlund54} first published in 1924.
In computer science, this representation was popularized by Knuth~\cite{Knuth98a},
who attributed it to S.O.~Rice, so that
it also came to be known as ``Rice's method''; see~\cite{FlSe95}
for  a review. 

\begin{lemma} Let $\phi(s)$ be holomorphic in the half-plane 
$\Re(s)\ge n_0-\frac12$. Then the finite differences of the sequence
$(\phi(k))$ admit the integral representation
\begin{equation}\label{norint}
\sum_{k=n_0}^n \binom{n}{k}(-1)^k\phi(k)=\frac{(-1)^{n}}{2\pi i}\int_C 
\phi(s) \frac{n!}{s(s-1)\cdots(s-n)}\, ds,
\end{equation}
where the contour of integration~$C$ encircles the integers $\{n_0,\ldots,n\}$
in a positive direction and is contained in~$\Re(s)\ge n_0-\frac12$.
\end{lemma}
\begin{proof}
The integral on the right of~\eqref{norint} is the sum of its residues at~$s=n_0,\ldots,n$,
which precisely equals the sum on the left.
\end{proof}

An immediate consequence is the following representation for the 
differences of zeta values ($\delta_n$ in~\eqref{deldef}):
\begin{equation}\label{norbn}
\delta_n\equiv \sum_{k=2}^n \binom{n}{k}(-1)^k\zeta(k)
=\frac{(-1)^{n-1}}{2\pi i}\int_{3/2-i\infty}^{3/2+i\infty} 
\zeta(s) \frac{n!}{s(s-1)\cdots(s-n)}\, ds. 
\end{equation}
(Start from a small rectangle encircling $\{2,\ldots,n\}$, then 
extend it to a long rectangle with horizontal sides at $\pm i\infty$,
and finally push the right vertical side to $+\infty$.
The contributions relative to 
% the horizontal sizes and the right vertical side 
all but the left vertical side vanish,
since $\zeta(s)$ remains bounded in modulus by~$\zeta(\frac32)$.)

Since~$b_n$ is $\delta_n$ plus a correction term (see Equation~\eqref{eq2}),
the first step is to move the line of integration further to the left.
It is well known that the Riemann zeta function is of finite order in any right half-plane,
that is, $|\zeta(s)|=O(|s|^A)$ uniformly as $|s|\to\infty$, for some~$A$ depending on the
half-plane under consideration~\cite{Titchmarsh86}.
As a consequence, the integral of~\eqref{norbn} remains 
convergent, when taken along any vertical line left of~0,
as soon as~$n$ is large enough. Under these conditions, it is possible
to replace the line of integration $\Re(s)=\frac32$ by the line~$\Re(s)=-\frac12$, upon
taking into account the residues of a double pole at $s=1$ and a simple
pole at $s=0$.
We find in this way
\[
\delta_n=(-1)^{n-1} (R_1+R_0)+\frac{(-1)^{n-1}}{2\pi i}\int_{-1/2-i\infty}^{-1/2+i\infty} 
\zeta(s) \frac{n!}{s(s-1)\cdots(s-n)}\, ds, \]
where, as shown by a routine calculation:
\[
(-1)^n R_0=-\frac12,
\qquad
(-1)^nR_1=-n(1-\gamma-H_{n-1}).
\]
The residues thus compensate exactly 
for the difference between~$\delta_n$ and~$b_n$, so that
\begin{equation}\label{norbn2}
b_n=
\frac{(-1)^{n-1}}{2\pi i}\int_{-1/2-i\infty}^{-1/2+i\infty} 
\zeta(s) \frac{n!}{s(s-1)\cdots(s-n)}\, ds.
\end{equation}

\section{Saddle point analysis of zeta differences}\label{sadzeta-sec}

The integrand in~\eqref{sadzeta-sec} is free of 
singularities on the left-hand side, and is thus amenable to evaluation
by the saddle point (or steepest-descent) method.  This evaluation 
is the main topic of this section, and culminates with the derivation of
one of the principal results of this paper, equation~\eqref{asympform}.

It is convenient to perform a change of variables $s\mapsto -s$.
The new integrand involves $\zeta(-s)$, which transforms by virtue of the 
functional equation 
of the zeta function,
\begin{equation}\label{funzeta}
\zeta(1-s)=(2\pi)^{-s}2\cos\left(\frac{s\pi}{2}\right)\Gamma(s)\zeta(s).
\end{equation}
A combination of~\eqref{norbn2} and~\eqref{funzeta} then gives
\begin{equation}\label{mainint}
b_n = -\frac{1}{\pi i} 
\int_{1/2-i\infty}^{1/2+i\infty} 
(2\pi)^{-s-1} \sin\left(\frac{\pi s}{2}\right)\zeta(1+s)
\frac{n!\,\Gamma(1+s)}{s(s+1)\cdots(s+n)}\, ds,
\end{equation}
which is the starting point of our asymptotic analysis.

The integral representation~\eqref{mainint} has several noticeable features. 
First, the integrand has no singularity at all in $\Re(s)\ge\frac12$
and it decays fast enough towards $\pm i \infty$,
which means that one can freely choose the abscissa $c$ (with $c\geq\frac12$)
in the representation 
\begin{equation}\label{mainint2}
b_n = -\frac{1}{\pi i} 
\int_{c-i\infty}^{c+i\infty} 
(2\pi)^{-s-1} \sin\left(\frac{\pi s}{2}\right)\zeta(1+s)
\frac{n!\, \Gamma(s+1)}{s(s+1)\cdots(s+n)}\, ds.
\end{equation}
The very absence of singularities calls for an application 
of the saddle  point method.

The factor $\zeta(1+s)$ remains bounded in modulus by a constant,
and is in fact barely distinguishable from~1, as $\Re(s)$ increases,
since
\begin{equation}\label{zetapp}
\zeta(s)=1+\mathcal{O}\left(2^{-\Re(s)}\right), \qquad \Re(s)\ge\frac32.
\end{equation}
Also, for large $|s|$, the complex version of Stirling's formula 
applies:
\begin{equation}\label{stirapp}
\Gamma(1+s) = s^s e^{-s}\sqrt{2\pi s}\left(1+\mathcal{O}(|s|^{-1})\right),\qquad \Re(s)\ge0.
\end{equation}
Finally, the sine factor increases exponentially along vertical lines:
one has
\begin{equation}\label{sinapp}
2i\sin\frac{\pi s}{2} =-\exp\left(-i\frac{\pi s}{2}\right)
+\mathcal{O}\left(e^{-\pi \Im(s)/2}\right), \qquad \Im(s)\ge0,
\end{equation}
with a conjugate approximation holding for $\Im(s)<0$.
% (The fact decay at complex infinity of $\Gamma(s)$ then 
% compensates for the growth of the sine factor, so that, ultimately, the integrand 
% decreases polynomially fast, due to the rising factorial in the denominator, roughly
% like $|s|^{-n}$.)

In anticipation of applying saddle-point methods, 
the approximations~\eqref{zetapp}, \eqref{stirapp}, and~\eqref{sinapp}
then suggest the function $e^{\omega(s)}$
 as a simplified model of the integrand in the upper half-plane, where
\begin{equation}\label{omdef}
\omega(s)=-s\log(2\pi)-i\frac{\pi s}{2}+\log\frac{n! \Gamma(s)^2}{\Gamma(s+n)}.
\end{equation}
We shall demonstrate shortly that the location of the appropriate saddle 
points in the complex plane scale as $\sqrt{n}$, which may be confirmed 
by numerical experiments. Therefore, in performing an asymptotic analysis,
it is appropriate to perform a change of variable $s=x\sqrt{n}$, 
and expand in descending powers of $n$, presuming $x$ to be approximately constant.
We find, uniformly for $x$ in any compact region of $\Re(x)>0$, $\Im(x)>0$:
\begin{equation}\label{om012}
\left\{\begin{array}{lll}
\omega(x\sqrt{n})&=&\ds x\sqrt{n}\left[2\log x-2-\log(2\pi)-\frac12i\pi \right]+\frac12\log n
\\
&&\ds \qquad -\log x +\log(2\pi)-\frac12x^2+\mathcal{O}(n^{-1/2})
\\
\omega'(x\sqrt{n})&=& \ds \left[-\log(2\pi) -i\frac{\pi}{2}+2\log x\right]
-\left(x+\frac1x\right)\frac{1}{\sqrt{n}}+\mathcal{O}(n^{-1}).
\\
\omega''(x\sqrt{n})&=&\ds \frac{2}{x\sqrt{n}}+\mathcal{O}(n^{-1}).
\end{array}\right. 
\end{equation}
(The symbolic manipulation system {\sc Maple} is a great help in such computations.)
% Also, at $s=x\sqrt{n}$, one has 
% \begin{equation}\label{om2}
% \begin{array}{lll}
% \log(n!\omega(x\sqrt{n})&=&\ds \sqrt{n}\left[2x\log x-2x-x\log(2\pi)-\frac12i\pi x\right]
% \\
% &=&\ds \quad -\log x -\log\sqrt{n}+\log(2\pi)-\frac12x^2+O(n^{-1/2}).
% \end{array}
% \end{equation}

From the second line of~\eqref{om012}, an approximate root of $\omega'(s)$
is obtained by choosing the particular value~$x_0$ of $x$ that cancels 
$\omega'$ to main asymptotic order:
\begin{equation}\label{x0}
x_0=e^{i\pi/4}\sqrt{2\pi}.
\end{equation}
This corresponds to the following value for $s$,
\begin{equation}\label{sad0}
\sigma\equiv \sigma(n)= x_0\sqrt{n} = % e^{i\pi/4} \sqrt{2\pi n} =
(1+i)\sqrt{\pi n},
\end{equation}
which is thus also an approximate saddle point for $e^{\omega(s)}$.
The substitution of this value given the first line of~\eqref{om012} then leads to
\begin{equation}\label{appsad}
\exp\left(\omega(\sigma(n))\right)
=\exp\left(2i\sqrt{\pi n}\right) \cdot \exp\left(-2\sqrt{\pi n}\right)\cdot \Pi(n),
\end{equation}
where $\Pi$ is an
unspecified factor of at most polynomial growth. 
By using a suitable contour that passes though~$\sigma(n)$,
we thus expect the quantity in~\eqref{appsad} to be an approximation
(up to polynomial factors again) of~$b_n$. This back-of-the-envelope calculation
does predict  the exponential decay of~$b_n$ as $\exp\left(-3.54490\sqrt{n}\right)$,
in a way consistent with numerical data, while 
the fluctuations,
$
\sin\left(2\sqrt{\pi n}+\mathcal{O}(1)\right),
$
are seen to be in stunning agreement with the empirically obtained formula~\eqref{conj}.

% Equipped with~\eqref{norbn2}, we propose to prove:
% 
% \begin{theorem} The $b_n$ satisfy the asymptotic expansion blabla.

\begin{figure}

\begin{center}
\begin{footnotesize}
\setlength{\unitlength}{0.6truecm}
\begin{picture}(6,10)(-1,-5)
\thicklines
\put(-1,0){\line(1,0){6}}
\put(0,-5){\line(0,1){10}}
\put(0,0.2){~$0$}
\put(1,4.2){$\Re(s)=c_1\sqrt{n}$}
\put(3,0.2){$\Re(s)=c_2\sqrt{n}$}
%%% begin upper diagram
\put(2,2){\circle*{0.2}}
\put(2,2){$\quad\ds \sigma=e^{i\pi/4}\sqrt{2\pi n}$}
\put(1,3){\line(1,-1){2}}
% \put(1,3){\line(2,-5){2}}
\put(1,3){\line(0,1){2}}
\put(3,1){\line(0,-1){1}}
\put(2,2){\vector(1,-1){0.5}}
\put(2,2){\vector(-1,1){0.5}}
\thinlines
\put(1,3){\line(0,-1){3}}
%%% end  upper diagram
%%% begin lower diagram
\thicklines
\put(2,-2){\circle*{0.2}}
\put(2,-2){$\quad\ds \overline\sigma=e^{-i\pi/4}\sqrt{2\pi n}$}
\put(1,-3){\line(1,1){2}}
\put(1,-3){\line(0,-1){2}}
\put(3,-1){\line(0,1){1}}
\put(2,-2){\vector(1,1){0.5}}
\put(2,-2){\vector(-1,-1){0.5}}
\thinlines
\put(1,-3){\line(0,1){3}}
%%% end  lower diagram
\end{picture}
\end{footnotesize}
\qquad
\Img{7.5}{plotc.jpg}
\end{center}
\caption{\label{sad-fig}
Left: The saddle point contour used for estimating $b_n$.
The arrows point at the directions of steepest descent
from the saddle points.
Right: the landscape of the logarithm of the modulus of the integrand
in the representation of~$b_n$ for $n=10$.}

\end{figure}

\smallskip
We  must   now fix the   contour  of  integration  and  provide  final
approximations. The   contour  adopted  (Figure~\ref{sad-fig})    goes
through the saddle  point $\sigma=\sigma(n)$ and symmetrically  through
its  complex conjugate $\overline\sigma=\overline{\sigma(n)}$.  In the
upper half-plane, it traverses $\sigma(n)$ along  a line  of steepest
descent   whose   direction, as   determined    from the  argument  of
$\omega''(\sigma)$,  is at an    angle of $\frac{5\pi}{8}$    with the
horizontal  axis.  The contour   also  includes parts of two  vertical
lines of  respective abscissae $\Re(s)=c_1\sqrt{n}$ and~$c_2\sqrt{n}$,
where
\[
0<c_1<\sqrt{\pi}<c_2<2\sqrt{\pi}.
\] 

The  choice of the abscissae, $c_1$  and $c_2$, is
not critical (it is even possible to adapt the analysis to $c_1=c_2=\sqrt{\pi}$).
One verifies  easily, from crude approximations, that
the contributions arising from the vertical parts of the contour are
$\mathcal{O}(e^{-L_0\sqrt{n}})$,    for   some $L_0>2\sqrt{\pi}$,   i.e.,   they  are
exponentially small in the  scale of the problem:
\begin{equation}\label{vert}
\int_{\operatorname{vertical}} = \mathcal{O}\left(e^{-L_0\sqrt{n}}\right)\qquad L_0>2\sqrt{\pi}.
\end{equation}


The slanted part of
the contour is such that all the estimates of~\eqref{om012} apply. 
The scale of the problem is dictated by the value of $\omega''(\sigma)$,
 which is of order $\mathcal{O}(n^{-1/2})$. This indicates that the ``second order'' scaling 
to be adopted is~$n^{1/4}$. Accordingly, we set
\begin{equation}\label{cregion}
s=(1+i)\sqrt{\pi n}+e^{5i\pi/8}yn^{1/4}.
\end{equation}
% Equivalently, we let $x$ vary slightly 
% around $e^{i\pi/4}\sqrt{2\pi}$ by amounts proportional to $n^{-1/4}$.
Define the \emph{central region} of the slanted part of the contour by the condition
that $|y|\le \log^2 n$.
Upon slightly varying the value of~$x$ around~$x_0$, one verifies from~\eqref{om012} that,
for large~$n$, the quantity
\[
\Re\left(\frac{1}{\sqrt{n}}
\omega\left(x_0\sqrt{n}+e^{5i\pi/8}t\sqrt{n}\right)\right)
\]
is an upward concave function of~$t$ near~$t=0$. There results,
in the complement of the central part, $|y|\ge \log^2n$,
the approximation
\[
\left|\exp\left(\omega\left(x_0\sqrt{n}+e^{5i\pi/8}yn^{1/4}\right)\right)\right|
< e^{\omega(x_0\sqrt{n})}
\cdot \exp\left(-L_1 \log^2 n\right),\qquad L_1>0.
\]
Figuratively:
\begin{equation}\label{cent}
\int_{\operatorname{slanted}}=\int_{\operatorname{central}}+\mathcal{O}\left(\exp\left(-L_1 \log^2 n\right)\right).
\end{equation}
Thus, from~\eqref{vert}  and~\eqref{cent},  only the central   part of the
slanted region matters asymptotically. This applies to $e^{\omega(s)}$ but also
to the full integrand of the representation~\eqref{mainint2} of $b_n$,
given the approximations~\eqref{zetapp}--\eqref{om012}.

\smallskip

We are finally ready to reap the crop. Take the integral representation
of~\eqref{mainint2} with the contour deformed as indicated in Figure~\ref{sad-fig} and
let $b_n^{+}$ be the contribution arising from the upper half-plane, to
the effect that
\begin{equation}\label{conjug}
b_n=2\Re(b_n^+),
\end{equation}
by conjugacy. In the central region, 
\[
s=x_0\sqrt{n}+e^{5i\pi/8}yn^{1/4},
\]
the integrand of~\eqref{mainint2} becomes
\begin{equation}\label{phis}
\left(-\frac{1}{\pi i}\right)\cdot 
(2\pi)^{-1}\cdot 
\left(-\frac{1}{2i}\right)\cdot
 \left(1+\mathcal{O}(2^{-\sqrt{n}})\right)\cdot 
\frac{x_0}{n} \cdot e^{\omega(x_0\sqrt{n})} \cdot
e^{-y^2/\sqrt{2\pi}}\left(1+\mathcal{O}(\frac{1}{\sqrt{n}}\right).
\end{equation}
The various factors found there (compare~\eqref{mainint2} to $e^{\omega(s)}$
with $\omega(s)$ defined in~\eqref{omdef})
are in sequence: the Cauchy integral prefactor; 
the correction $(2\pi)^{-1}$ to the functional equation of Riemann zeta;
the factor $-1/(2i)$ relating the sine to its
exponential approximation; the approximation of Riemann zeta;
the correction $s/(s+n)$ of the Gamma factors; 
the main term $e^{\omega(\sigma)}$;
the anticipated local Gaussian approximation;
the errors resulting from approximations~\eqref{zetapp}--\eqref{om012},
which are of relative order $O(n^{-1/2})$. Upon completing the tails of the 
integral and neglecting exponentially small corrections,
we get
\begin{equation}\label{asy0}
b_n^+ =K_0 e^{\omega(x_0\sqrt{n})}
\frac{x_0}{\sqrt{n}}
\int_{-\infty}^{+\infty} e^{-y^2/\sqrt{2\pi}}\,dy
\cdot\left(e^{5i\pi/8}n^{1/4}\right)\left(1+\mathcal{O}\left(\frac{1}{\sqrt{n}}\right)\right),
\end{equation}
where~$K_0$    is the constant  factor  of~\eqref{phis}, while the factor
following   the integral   translates the   change   of   variables:
$ds=e^{5i\pi/8}n^{1/4}dy$. 

The  asymptotic form of~$b_n$ is now completely
determined by~\eqref{conjug} and~\eqref{asy0}.
We have obtained:

\begin{figure}
\begin{center}
\Img{7}{comparapp.jpg}
\end{center}
\caption{\label{comparapp-fig} A comparative plot of $b_n$ and 
the main term of its approximation~\eqref{thmz},
both multiplied by $e^{-2\sqrt{\pi n}}n^{-1/4}$, for $n=5\,.\,.\,500$.}
\end{figure}

\begin{theorem} \label{zetacoeff-thm}
The Newton coefficient $b_n$ of $\zeta(s)-1/(s-1)$ defined in~\eqref{eq2}
satisfies
\begin{equation}\label{thmz}
b_n = \left( \frac{2n}{\pi}\right)^{1/4}
e^{-2\sqrt{\pi n}}\cos\left(2\sqrt{\pi n}-\frac{5\pi}{8}\right)
+\mathcal{O}\left(e^{-2\sqrt{\pi n}}n^{-1/4}\right).
% \qquad
% K:=2^{3/4}\pi^{-1/4}.
\end{equation}
\end{theorem}

The agreement between asymptotic and exact values is quite good,
even for small values of~$n$ (Figure~\ref{comparapp-fig}).

\smallskip

The foregoing developments justify a posteriori the application of the saddle point formula to the N\"orlund Rice
integral representation~\eqref{mainint2} of zeta value differences. This formula
reads
\begin{equation}\label{sadfor}
\int e^{-Nf(x)} \, dx =\sqrt{\frac{2\pi}{N|f''(x_0)|}}
e^{-N f(x_0)}\left(1+\mathcal{O}\left(\frac{1}{N}\right)\right).
\end{equation}
Here the analytic function $f(x)$ should be such that $|f(x)|$ has a saddle-point at~$x_0$,
that is, $f'(x_0)=0$ and $f''(x_0)$ is the second derivative of~$f$ at the saddle point.
In the case of differences of zeta values, 
the appropriate scaling parameter is $s=x\sqrt{n}$ corresponding to $N=\sqrt{n}$, and the
the function~$f$ is
\[
f(x)=\lim_{n\to\infty} \frac{1}{\sqrt{n}}\omega\left(x\sqrt{n}\right),
\]
up to smaller order corrections that can be treated as constants in the range of the saddle point.

\section{Convergence of the Newton series of zeta}\label{conv-sec}

The fact that the coefficients $b_n$ decay to zero faster than any polynomial in~$1/n$ 
implies that the Newton series
\begin{equation}\label{newtonz}
\Phi(s)=\sum_{n=0}^\infty (-1)^n b_n \binom{s}{n},
\end{equation}
with $b_n$ given by~\eqref{eq2},
converges throughout the complex plane, and consequently defines an entire function.
Set $Z(s):=\zeta(s)-1/(s-1)$ with~$Z(1)=\gamma$. We have, by construction $\Phi(s)=Z(s)$
at $s=0,1,2\ldots\,$, but the relation between $\Phi$ and $Z$ at other points is still unclear.

\begin{corollary}\label{newton-cor} The Newton series of~\eqref{newtonz} 
is a convergent representation of the function $\zeta(s)-1/(s-1)$
valid at all points $s\in\C$.
\end{corollary}
\begin{proof} 
Here  is our favorite  proof.  A   classic theorem  of Carlson  (for a
discussion        and      a     proof,     see,      e.g.,    Hardy's
Lectures~\cite[pp.~188-191]{Hardy78}          or          Titchmarsh's
treatise~\cite[\S5.81]{Titchmarsh39})  says the following:
\emph{Assume that $(i)$~$g(s)$ is
analytic and such that
\[
\left|g(s)\right|<C^{A|s|},
\]
where $A<\pi$, in the right half-plane of complex values of~$s$,
and $(ii)$~$g(0)=g(1)=\ldots=0$. Then $g(s)$ vanishes identically.}


To complete the proof, it  suffices to apply  Carlson's theorem to the
difference    $g(s)=\Phi(s+2)-Z(s+2)$.
Condition~$(ii)$ is    satisfied  by   construction  of    the  Newton
series. Condition~$(i)$ results from  the fact that $Z(s+2)$ is $\mathcal{O}(1)$
while  a general bound  due to N\"orlund 
(Equation~(58) of~\cite[p.~228]{Norlund54})
and valid for all convergent Newton series
asserts that $|\Phi(s+2)|$  is of 
growth at most $e^{\frac{\pi}{2}|s|}$, throughout $\Re(s)>-\frac12$.
% I struggled quite a bit to find a direct elementary proof instead.
% Please don't discard what follows for the moment!
% 
% Regarding the promised crude bounds,
% use $f(x)\ll g(x)$ to express that $f(x)\le M g(x)$ for some absolute constant~$M$. We have,
% with $S=|s|$:
% \[
% \begin{array}{lll}
% |\Phi(s)|& \ll & \ds \sum_{n\ge0}e^{-\sqrt{n}}
% \left|\binom}{s}{n}\right| \ll 
% \left(\sum_{0\le n \le 2S}\binom{S+n-1}{n}\right)+
% \left(\sum_{2S<n} e^{-\sqrt{n}}
% \left|\binom}{s}{n}\right|\right) \\
% &\ll& \ds e^{2S}+\sum_{2S<n} e^{-\sqrt{n}}
% \left|\binom}{s}{n}\right|.
% \end{array}
% \]
% Next, for $\Re(s)>0$ and $n>2|s|$, simple geometry shows that
% \[
% \left|\binom}{s}{n}\right| \le \left|\binom}{in/2}{n}\right|
% 
% 
% \binom{S+n-1}{n}\ll\sum_{n\ge0}e^{-\sqrt{n}}\frac{(S+n)^n}{n!}
% \ll \sum_{n\ge0}e^{-\sqrt{n}}\frac{n^n}{n!}\left(1+\frac{S}{n}\right)^n\\
% &\ds \ll& e^{n\log(1+S/n)}\ll e^S.
% \end{array}
% \]
\end{proof}

An alternative proof can be given starting from a contour integral representation for the remainder of 
a general Newton series given~\cite[p.~223]{Norlund54}. In a short note, B\'aez-Duarte~\cite{Baez03}
justified a similar looking Newton series representation of the zeta function due to Ma\'slanka---however
his bounds on the Newton coefficients are less precise than ours and his arguments (based on
a doubly indexed sequence of polynomials) seem to be somewhat problem-specific.




\section{Dirichlet $L$-functions}\label{dir-sec}

The methods employed to deal with differences of zeta values have a
more general scope, and we may reasonably expect them to be applicable to
other kinds of Dirichlet series.
%  which are of the form
% \[
% \phi(s)=\sum_{n\ge1} \frac{f_n}{n^s},
% \]
% after possible desingularization. 
Such is indeed the case
for any Dirichlet $L$-function,
\[
L(\chi,s)=\sum_{n=1}^\infty \frac{\chi(n)}{n^s},
\]
where $\chi$ is a multiplicative character 
of some period~$k$, that is,
for all integers~$m,n$, one has:
$\chi(n+k)=\chi(n)$,  $\chi(mn)=\chi(m)\chi(n)$, $\chi(1)=1$, and
$\chi(n)=$ whenever~$\gcd(n,k)\not=1$. 


% This section defines the Dirichlet $L$-functions, and sets up 
% the asymptotic analysis of their Newton series. The development
% parallels the analysis performed for the Riemann zeta, and so
% is given in abbreviated form. The results are similar in nature,
% and indeed, the result for the Riemann zeta can be retrieved as 
% a special case.  Thus, an effort will be made to focus on how these
% sums differ.
% 
% The Dirichlet $L$-functions\cite{Apostol76,Davenport80} are defined
% in terms of the Dirichlet characters, which are group representation
% characters of the cyclic group. 
% % They play an important role in number
% % theory, and the Riemann hypothesis generalizes to the \emph{L}-functions.
% The Dirichlet characters are multiplicative functions, and are periodic
% modulo $k$. That is, a character $\chi(n)$ is an arithmetic function
% of an integer $n$, with period $k$, such that $\chi(n+k)=\chi(n)$.
% A character is multiplicative, in that $\chi(mn)=\chi(m)\chi(n)$
% for all integers $m,n$. Furthermore, one has that $\chi(1)=1$ and
% $\chi(n)=0$ whenever $\gcd(n,k)\ne1$. The \emph{L}-function associated
% with the character $\chi$ is defined as \begin{equation}
% L(\chi,s)=\sum_{n=1}^{\infty}\frac{\chi(n)}{n^{s}}\label{eq:}\end{equation}


Let $\zeta(s,q)$ be the Hurwitz
zeta function defined by
\begin{equation}\label{defhur}
\zeta(s,q)=\sum_{n=0}^{\infty}\frac{1}{(n+q)^{s}}\end{equation}
 Any $L$-function may be represented  as a combination of Hurwitz zeta
\begin{equation}
L(\chi,s)=\frac{1}{k^{s}}\sum_{m=1}^{k}\chi(m)\zeta\left(s,\frac{m}{k}\right)\label{eq:L-Hurwitz}\end{equation}
where $k$ is the period of $\chi$.
%  and $\zeta(s,q)$ is the Hurwitz
% zeta function, given by \begin{equation}
% \zeta(s,q)=\sum_{n=0}^{\infty}\frac{1}{(n+q)^{s}}\label{eq:}\end{equation}
% Thus, the study of the analytic properties of the \emph{L}-functions
% can be partially unified through the study of the Hurwitz zeta function.
In particular, the coefficients of the Newton series  for $L(\chi,s)$ 
% 
% \section{Forward differences}
% 
% In analogy to the forward differences of the Riemann
% zeta function, 
% the coefficients of the Newton series  for $L(\chi,s)$ 
% being given by
% \begin{equation*}
% L_{n}=\sum_{\ell=2}^{n}(-1)^{\ell}
% \binom{n}{\ell},
% L(\chi,\ell)\label{eq:L-coeff}\end{equation*}
%  Because of the relation~\eqref{eq:L-Hurwitz} connecting the Hurwitz
% zeta function to the L-function, 
are simple linear combinations of the quantities
\begin{equation}\label{defA}
A_{n}(m,k)=\sum_{\ell=2}^{n}
\binom{n}{\ell}
(-1)^{\ell}\frac{\zeta\left(\ell,\frac{m}{k}\right)}{k^{\ell}},
\end{equation}
which we adopt as our fundamental object of study.
% since 
% \begin{equation*}
% L_{n}=\sum_{m=1}^{k}\chi(m)\, A_{n}(m,k)
% \end{equation*}

\begin{theorem}\label{L-thm} The differences of Hurwitz zeta values, $A_n(m,k)$
defined by~\eqref{defA}, satisfy the estimate
\begin{equation}\label{A1}
A_{n}(m,k)=\left(\frac{m}{k}-\frac{1}{2}\right)
-\frac{n}{k}\left[\psi\left(\frac{m}{k}\right)+\ln k+1-H_{n-1}\right]
+a_{n}(m,k)
\end{equation}
where the $a_n(m,k)$ are exponentially small:
\begin{equation}\label{A2}
\begin{array}{lll}
a_{n}(m,k)&=&\ds \frac{1}{k}\left(\frac{2n}{\pi k}\right)^{1/4}
\exp\left(-\sqrt{\frac{4\pi n}{k}}\right)
\cos\left(
\sqrt{\frac{4\pi n}{k}} -\frac{5\pi}{8} -\frac{2\pi m}{k} \right)
\\
&& {} \ds +\mathcal{O}\left(n^{-1/4}e^{-2\sqrt{\pi n/k}}\right).
\end{array}
\end{equation}
\end{theorem}

The previous results for Riemann zeta may be regained by setting 
$m=k=1$, so that $\delta_n=A_n(1,1)$ and $b_n=a_n(1,1)$.


\begin{proof}
 Converting the sum to the N\"orlund-Rice integral, and extending the
contour to infinity, one obtains 
\begin{equation}
A_{n}(m,k)=\frac{(-1)^{n}}{2\pi i}\, n!\,
\int_{\frac{3}{2}-i\infty}^{\frac{3}{2}+i\infty}
\frac{\zeta\left(s,\frac{m}{k}\right)}{k^{s}s(s-1)\cdots(s-n)}\, ds
\end{equation}
 Moving the contour to the left, one encounters a single pole at $s=0$
and a double pole at $s=1$. The residue of the pole at $s=0$ is
\begin{equation*}
\mbox{Res}(s=0)=\zeta\left(0,\frac{m}{k}\right)=\frac{1}{2}-\frac{m}{k}.
\end{equation*}
(See~\cite[p.~271]{WhWa27} for this evaluation.)
%  where one has the curious identity in the form of a multiplication
% theorem for the digamma function:
% \begin{equation*}
% \zeta\left(0,\frac{m}{k}\right)
% =\frac{-1}{\pi k}\sum_{p=1}^{k}\sin\left(\frac{2\pi pm}{k}\right)
% \psi\left(\frac{p}{k}\right)
% =-B_{1}\left(\frac{m}{k}\right)
% =\frac{1}{2}-\frac{m}{k}
% \end{equation}
%  Here, $\psi$ is the digamma function and $B_{1}$is the Bernoulli
% polynomial of order 1. 
The double pole at $s=1$ evaluates to
\begin{equation*}
\mbox{Res}(s=1)=\frac{n}{k}
\left[\psi\left(\frac{m}{k}\right)+\ln k+1-H_{n-1}\right]
\end{equation*}
Combining these, one obtains~\eqref{A1}
% \begin{equation*}
% A_{n}(m,k)=\left(\frac{m}{k}-\frac{1}{2}\right)
% -\frac{n}{k}\left[\psi\left(\frac{m}{k}\right)+\ln k+1-H_{n-1}\right]
% +a_{n}(m,k),
% \end{equation*}
where the $a_n$ are given by
\begin{equation}\label{defa}
a_{n}(m,k)=\frac{(-1)^{n}}{2\pi i}\, n!\,
\int_{-\frac{1}{2}-i\infty}^{-\frac{1}{2}+i\infty}
\frac{\zeta\left(s,\frac{m}{k}\right)}{k^{s}s(s-1)\cdots(s-n)}\, ds.
\end{equation}

% The previous results for the Riemann zeta may be regained by setting 
% $m=k=1$, so that $\delta_n=A_n(1,1)$ and $b_n=a_n(1,1)$.  
As before,
the $a_n(m,k)$ have the remarkable property of being exponentially
small; that is, 
% \begin{equation}
$ a_{n}(m,k)=\mathcal{O}\left(e^{-\sqrt{Kn}}\right)$,
% \end{equation}
 for a constant $K$ that only depends on~$k$. 
% The next section develops an
% explicit asymptotic form for this term.
% \section{Saddle-point analysis of $L$-function differences}
The precise behavior of the exponentially small term may be obtained 
by using the same saddle-point analysis given in the previous
sections. Again, its application here is abbreviated, as there
are no substantial differences in the course of the derivations.

The term $a_{n}(m,k)$ is represented by the integral of~\eqref{defa}.
% \begin{equation}
% a_{n}(m,k)=\frac{(-1)^{n}}{2\pi i}\, n!\,
% \int_{-\frac{1}{2}-i\infty}^{-\frac{1}{2}+i\infty}
% \frac{\zeta\left(s,\frac{m}{k}\right)}{k^{s}s(s-1)\cdots(s-n)}\, ds
% \label{eq:little-a-integral}\end{equation}
% which resulted from shifting the integration contour past the poles.
At this point, the functional equation for the Hurwitz zeta may be
applied. This equation is 
\begin{equation}
\zeta\left(1-s,\frac{m}{k}\right)
=\frac{2\Gamma(s)}{(2\pi k)^{s}}
\sum_{p=1}^{k}\cos\left(\frac{\pi s}{2}-\frac{2\pi pm}{k}\right)
\zeta\left(s,\frac{p}{k}\right)
\end{equation}
 This allows the integral to be expressed as a sum:
\begin{equation*}
a_{n}(m,k)=-\frac{2n!}{k\pi i}\sum_{p=1}^{k}
\int_{\frac{3}{2}-i\infty}^{\frac{3}{2}+i\infty}
\frac{1}{(2\pi)^{s}}\frac{\Gamma(s)\Gamma(s-1)}{\Gamma(s+n)}
\cos\left(\frac{\pi s}{2}-\frac{2\pi pm}{k}\right)
\zeta\left(s,\frac{p}{k}\right)\, ds
\end{equation*}
 It proves  convenient to pull the phase factor out of the
cosine part % ; we do this now, 
and write the integral as 
\begin{equation*}
\begin{array}{lll}
a_{n}(m,k) & = & \ds -\frac{n!}{k\pi i}
\sum_{p=1}^{k}\exp\left(i\frac{2\pi pm}{k}\right) \\
 & & \quad \ds
\int_{\frac{3}{2}-i\infty}^{\frac{3}{2}+i\infty}
\frac{1}{(2\pi)^{s}}\frac{\Gamma(s)\Gamma(s-1)}{\Gamma(s+n)}\,
\exp\left(-i\frac{\pi s}{2}\right)
\zeta\left(s,\frac{p}{k}\right)\, ds   +{\bf c.c.},
\end{array}
\label{eq:two-integrals}
\end{equation*}
 where ${\bf c.c.}$ (``\emph{complex conjugate}'') means that $i$ should be replaced by $-i$ in the
two exp parts.


% For large values of $n$, this integral may be evaluated by means
% of the saddle-point method.
% % , just like in the case of
% % differences of the Riemann zeta function.
% %  The saddle-point method, or method of
% % steepest descents, may be applied whenever the integrand can be approximated
% % by a sharply peaked Gaussian, as the above can be for large $n$.
% % More precisely, The saddle-point theorem states that \begin{equation}
% % \int e^{-Nf(x)}dx\approx\sqrt{\frac{2\pi}{N\left|f^{\,\prime\prime}(x_{0})\right|}}e^{-Nf(x_{0})}\left[1-\frac{f^{(4)}(x_{0})}{8N\left|f^{\,\prime\prime}(x_{0})\right|^{2}}+\cdots\right]\label{eq:}\end{equation}
% %  is an asymptotic expansion for large $N$. Here, the function $f$
% % is taken to have a local minimum at $x=x_{0}$ and $f^{\,\prime\prime}(x_{0})$
% % and $f^{(4)}(x_{0})$ are the second and fourth derivatives at the
% % local minimum. 
% % 
% In what follows, we dispense with analytic details---these would
% closely mimic what was done in Section~\ref{sadzeta-sec}---and directly proceed  
% from the saddle point formula~\eqref{sadfor}.
% 

To recast the equation~\eqref{eq:two-integrals} into the form needed
for the saddle point method, an asymptotic expansion of the
integrands needs to be made for large $n$. 
% After such an expansion,
% it is seen that the saddle point occurs at large values of $s$, and
% so an asymptotic expansion in large $s$ is warranted as well. 
% As
% it is confusing and laborious to simultaneously expand in two parameters,
% it is better to seek out an order parameter to couple the two. This
% may be done as follows. 
% One notes that 
% the integrands have a minimum,
% on the real $s$ axis, near $s=\sigma_{0}=\sqrt{\pi kn/p}$ and so
As before, the appropriate scaling parameter is $x=s/\sqrt{n}$,
and so one may perform a change of variable from $s$ to $x$. The asymptotic
expansion is then performed by holding $x$ constant, and taking $n$
large. Thus, one writes 
\begin{equation}\label{eq:saddle}
a_{n}(m,k) = -\frac{1}{k\pi i}\sum_{p=1}^{k}
\left[e^{i {2\pi pm}/{k}}
\int_{\sigma_{0}-i\infty}^{\sigma_{0}+i\infty}e^{\omega(x\sqrt{n})}dx\right.
\left.+e^{-i{2\pi pm}/{k}}
\int_{\sigma_{0}-i\infty}^{\sigma_{0}+i\infty}
e^{\overline{\omega}(x\sqrt{n})}dx\right].
\end{equation}
% \begin{equation}
% a_{n}(m,k) = -\frac{1}{k\pi i}\sum_{p=1}^{k}
% \left[\exp\left(i\frac{2\pi pm}{k}\right)
% \int_{\sigma_{0}-i\infty}^{\sigma_{0}+i\infty}e^{\omega(x\sqrt{n})}dx\right.
% \label{eq:saddle}
% %  &  & \qquad \qquad 
% \left.+\exp\left(-i\frac{2\pi pm}{k}\right)
% \int_{\sigma_{0}-i\infty}^{\sigma_{0}+i\infty}
% e^{\overline{\omega}(x\sqrt{n})}dx\right]
% \end{equation}
%  where $\overline{\omega}$ is the complex conjugate of $\omega$. 

Proceeding, one finds 
\begin{equation*}
\omega(s)=\log n!+\frac{1}{2}\log n
-s\log\left(\frac{2\pi p}{k}\right)-i\frac{\pi s}{2}
+\log\frac{\Gamma(s)\Gamma(s-1)}{\Gamma(s+n)}
+\mathcal{O}\left( \left(\frac{p}{k+p}\right)^s \right)
\end{equation*}
 where the approximation that 
$\log \zeta\left(s,p/k\right)=(k/p)^{s}+\mathcal{O}\left( (p/(k+p))^s \right)$
for large $s$ has been made. 
% We don't use this!
% More generally, one has \begin{equation}
% \log\zeta(s)=\sum_{n=2}^{\infty}\;\frac{\Lambda(n)}{n^{s}\log n}\label{eq:}\end{equation}
%  where $\Lambda(n)$ is the Mangoldt function. (XXX What about Hurwitz?)
% The asymptotic expansion for the Gamma function is given by the Stirling
% expansion~\eqref{stirapp}.
% Do we need Bernoulli numbers??
% \begin{equation}
% \log\Gamma(x)=\left(x-\frac{1}{2}\right)\log x-x+\frac{1}{2}\log2\pi+\sum_{j=1}^{\infty}\frac{B_{2j}}{2j(2j-1)x^{2j-1}}\label{eq:}\end{equation}
% and $B_{k}$ are the Bernoulli numbers. 
Expanding to $\mathcal{O}(1/\sqrt{n})$
and collecting terms, one obtains 
\begin{equation}\begin{array}{lll}
\omega(x\sqrt{n}) & = & \ds \frac{1}{2}\log n
-x\sqrt{n}\left[\log\frac{2\pi p}{k}+i\frac{\pi}{2}+2-2\log x\right] \\
 &  & \ds  {} +\log2\pi-2\log x-\frac{x^{2}}{2}+\mathcal{O}\left(n^{-1/2}\right) 
% &  & +\frac{1}{6x\sqrt{n}}\left[10+x^{2}\right]
%+\frac{1}{2n}\left[1-\frac{x^{2}}{2}-\frac{x^{4}}{6}
%+\frac{73}{72x^{2}}\right]
%+\mathcal{O}\left(n^{-3/2}\right).
\end{array}
\end{equation}
% \footnote{
%  LV: After substitution of p=k=1 in this formula, one of the log x 
% terms differs by a factor of two from that in equation \eqref{om012}. 
% This should be resolved.
% }

 The saddle point is obtained  by solving $\omega'(x\sqrt{n})=0$.  To
 lowest order, one  obtains $x_{0}=(1+i)\sqrt{\pi  p/k}$. To  use  the
 saddle-point  formula,  one needs
 $\omega''(x\sqrt{n})=2/x\sqrt{n}+\mathcal{O}(n^{-1})$.
 Substituting, one directly obtains 
\begin{equation}
\int_{\sigma_{0}-i\infty}^{\sigma_{0}+i\infty}
\!\!\! e^{\omega(x\sqrt{n})}dx 
=\left(\frac{2\pi^{3}pn}{k}\right)^{1/4}\!\!\!e^{i\pi/8}
\exp\!\left(-(1+i)\sqrt{\frac{4\pi pn}{k}}\right)\! {}
+\mathcal{O}\left(n^{-1/4}e^{-2\sqrt{\pi pn/k}}\right)
\end{equation}
 while the integral for $\overline{\omega}$ is the complex conjugate 
quantity (having a saddle point at the complex conjugate location). Inserting
this into equation \eqref{eq:saddle} gives a sum of contributions for $p=1,\ldots,k$,
of which, 
% \begin{equation}
% a_{n}(m,k)=\frac{1}{k}
% \sum_{p=1}^{k}\left(\frac{2np}{\pi k}\right)^{1/4}
% \exp\left(-\sqrt{\frac{4\pi pn}{k}}\right)
% \cos\left( \sqrt{\frac{4\pi pn}{k}} -\frac{5\pi}{8} -\frac{2\pi pm}{k} \right)
% +\mathcal{O}\left(n^{-1/4}e^{-2\sqrt{\pi pn/k}}\right)
% \label{eq:an-p}\end{equation}
for large $n$, only the $p=1$ term contributes significantly.
So, one has obtained the estimation~\eqref{A2} of the statement.
\end{proof}

% may write
% \begin{equation}
% a_{n}(m,k)=\frac{1}{k}\left(\frac{2n}{\pi k}\right)^{1/4}
% \exp\left(-\sqrt{\frac{4\pi n}{k}}\right)
% \cos\left(
% \sqrt{\frac{4\pi n}{k}} -\frac{5\pi}{8} -\frac{2\pi m}{k} \right)
% +\mathcal{O}\left(n^{-1/4}e^{-2\sqrt{\pi n/k}}\right)
% \end{equation}
%  which demonstrates the desired result: the terms $a_{n}$ are exponentially
% small. The previous result for the Riemann zeta, equation~\eqref{thmz}, 
% may be regained by substituting $m=k=1$, so that $b_n=a_n(1,1)$.

% xxxxxxxxxxxxxxxxxxxxxxxxxxxxxxxxxxxxxxxx
% \footnote{
%  LV: At this time, substituting m=k=1 into the above yields a formula 
% that differs by overall magnitude of sqrt(2) and also differs by a 
% phase of $\pi$ as compared to equation \eqref{thmz}.  My numerical
% work agrees with my calculation of $b_n$.
% }
% 
% The consistency of this asymptotic expansion may also be verified by
% applying the so-called "multiplication theorem" for the Hurwitz zeta,
% which states that
% 
% \begin{equation}
% \sum_{m=1}^{k}\zeta\left(s,\frac{m}{k}\right) =k^{s}\zeta(s)
% \end{equation}
% 
% From this, one may deduce both that 
% \begin{equation}
% \sum_{k=1}^{m}A_{n}(m,k)=\delta_n
% \end{equation}
%  and that 
% \begin{equation}
% \sum_{k=1}^{m}a_{n}(m,k)=b_{n}
% \label{eq:consistency}
% \end{equation}
%  In particular, the above must hold order by order in the asymptotic
% expansion. The correctness of the expansion given by equation~\eqref{eq:an-p}
% with regards to this identity may be readily verified by direct substitution.
% An important special case of equation \ref{eq:consistency} is the $k=2$
% relation 
% \begin{equation}
% a_{n}(2,2)=b_{n}-a_{n}(1,2)=\sum_{p=2}^{n}(-1)^{p}\binom{n}{p}\left(1-2^{-p}\right)\zeta(p)
% \end{equation}
% 
% which appears often in the literature {[}Coffey{]}.  It has the asymptotic expansion 
% \begin{equation}
% a_{n}(2,2)=\frac{1}{2}\sum_{p=1}^{2}\left(\frac{np}{\pi}\right)^{1/4}
% \exp\left(-\sqrt{2\pi np}\right) \cos\left(\sqrt{2\pi np}-\frac{5\pi}{8}\right)
% +\mathcal{O}\left(n^{-1/4}e^{-\sqrt{2\pi n}}\right)
% \end{equation}
% 
%                                                                                 
%                                                                                 
% 
% \section{Gauss sums}
% XXXX I'm not sure that this section is necessary; its off to the side,
% and just summarizes a few manipulations that are not particularly hard,
% and have no particular bearing on the main ideas. 
% ---------
% 
% We conclude by briefly returning to the structure of the Dirichlet
% L-functions. The L-function coefficients defined in equation \ref{eq:L-coeff}
% are now given by 
% \begin{equation}
% L_{n}=\sum_{m=1}^{k}\chi(m)A_{n}(m,k)
% \end{equation}
%  Writing 
% \begin{equation}
% A_{n}=B_{n}+a_{n}
% \end{equation}
%  so that $B_{n}(m,k)$represents the non-exponential part, one may
% state a few results. For the non-principal characters, one has $\sum_{m=1}^{k}\chi(m)=0$
% and thus, the first term simplifies to 
% \begin{equation}
% \sum_{m=1}^{k}\chi(m)B_{n}(m,k)
% =\frac{1}{k}\sum_{m=1}^{k}\chi(m)\left[m-n\psi\left(\frac{m}{k}\right)\right]
% \end{equation}
%  For the principal character $\chi_{1}$, one has $\sum_{m=1}^{k}\chi_{1}(m)=\varphi(k)$
% with $\varphi(k)$ the Euler totient function. Thus, for the principal
% character, one obtains 
% \begin{eqnarray}
% \sum_{m=1}^{k}\chi_{1}(m)B_{n}(m,k)
%  & = & -\varphi(k)\left[\frac12+\frac{n}{k}\left(\ln k+1-H_{n-1}\right)\right] \\
%  & & \qquad +\frac{1}{k}\sum_{m=1}^{k}\chi(m)\left[m-n\psi\left(\frac{m}{k}\right)\right]
% \end{eqnarray}
% 
% By contrast, the exponentially small term invokes a linear combination
% of Gauss sums. The Gauss sum associated with a character $\chi$ is
% \begin{equation}
% G(n,\chi)=\sum_{m\mbox{ mod}k}\chi(m)e^{2\pi imn/k}
% \end{equation}
%  and so, to leading order 
% \begin{eqnarray}
% \sum_{m=1}^{k}\chi(m)a_{n}(m,k) 
%  & = & \frac{1}{2k}\sum_{p=1}^k 
% \left(\frac{2pn}{\pi k}\right)^{1/4} \exp\left(-\sqrt{\frac{4\pi pn}{k}}\right) \\
%  &  & \quad \left[\exp i\left(\frac{5\pi}{8}-\sqrt{\frac{4\pi pn}{k}}\right)G(p,\chi) \right. \\
%  &  & \quad\qquad\left. -\exp-i\left(\frac{5\pi}{8}-\sqrt{\frac{4\pi pn}{k}}\right)G(-p,\chi)\right] \\
%  & & \quad +\mathcal{O} \left(n^{-1/4}e^{-2\sqrt{\pi n/k}}\right)
% \end{eqnarray}
% 
% The above expression simplifies slightly for the principle character,
% since one has the identities 
% \begin{equation}
% G\left(1,\chi_{1}\right)=\mu(k)
% \end{equation}
%  with $\mu(k)$ the M\"obius function and more generally, 
% \begin{equation}
% G\left(p,\chi_{1}\right)=
% \frac{\varphi(k)\mu\left(\frac{k}{(p,k)}\right)}{\varphi\left(\frac{k}{(p,k)}\right)}
% \end{equation}
                                                                                
% \section{wrapup}
\section{Perspective}\label{concl-sec}
\smallskip
The previous results serve to
% \footnote{%
% 	PF: We need a theorem statement summarizing both the principal and nonprincipal
% 	characters. LV: or drop the discussion of principle and non-principle characters.}
unify and make precise a number of estimates carried out in the literature by a diversity
of methods. For instance, the study of quantities arising in connection with Li's criterion
calls for estimating,  in the notations of~\eqref{defA},
\begin{equation}\label{coffeyS1}
A_n(2,2)=\sum_{k=2}^n \binom{n}{k}(-1)^k(1-2^{-k})\zeta(k).
\end{equation}
Coffey encountered this quantity (his~$S_1(n)$ in~\cite{Coffey05}),
and proved, by means of series rearrangements akin to~\eqref{simpdel}
used in conjunction with Euler-Maclaurin summation:
\begin{equation}\label{coffeyineq}
A_n(2,2) \ge \frac{n}{2}\log n+(\gamma-1)\frac{n}{2}+\frac12.
\end{equation}
Our analysis quantifies  the difference between the two quantities above as
being exponentially small and oscillating:
\begin{equation}
 a_{n}(2,2)=\frac{1}{2}\left(\frac{n}{\pi}\right)^{1/4}
\exp\left(-\sqrt{2\pi n}\right) \cos\left(\sqrt{2\pi n}-\frac{5\pi}{8}\right)
+\mathcal{O}\left(n^{-1/4}e^{-\sqrt{2\pi n}}\right)
\end{equation}

Another observation is that the combination of N\"orlund-Rice integrals
and saddle point estimates applies to many ``desingularized'' versions of 
the Riemann zeta function, like
\[
(1-2^{1-s})\zeta(s), \quad
(s-1)\zeta(s), \quad
\zeta(2s)-\frac{1}{2s-1},\quad
(2s-1)\zeta(2s).
\]
The first one is directly amenable to Theorem~\ref{L-thm}. The
Newton series involving $\zeta(2s)$ include Ma{\'s}lanka's expansion~\cite{Maslanka01}
(relative to $(2s-1)\zeta(2s)$)
and have a striking feature---their Newton coefficients are 
polynomials in~$\pi$ with rational coefficients. 
For a function like $\zeta(2s)-1/(2s-1)$ where the polar part is subtracted,
the exponential smallness of the coefficients
then has the peculiar feature of providing near identities that connect $\pi$ and
Euler's constant~$\gamma$.


\smallskip

The N\"orlund integrals are also of interest in the context of 
differences of inverse zeta values, for which curious relations
with the Riemann hypothesis have been noticed by Flajolet--Vall\'ee~\cite{FlVa00} and
independently by B\'aez-Duarte~\cite{Baez03b}. Consider the
typical quantity
\begin{equation}\label{invz}
d_n=\sum_{k=2}^n \binom{n}{k}(-1)^k \frac{1}{\zeta(k)},
\end{equation}
which arises as coefficient in the Newton series representation of $1/\zeta(s)$. 
Its asymptotic analysis can be approached by means of a N\"orlund-Rice representation
as noted by the authors of~\cite{FlVa00} and more recently by Ma\'slanka in~\cite{Maslanka06},
in related contexts.
We have:
\begin{theorem} The differences of inverse zeta values $d_n$ defined by~\eqref{invz} are such that
the following two assertions are equivalent: 
\begin{itemize}
\item[]{\bf FVBD  Hypothesis (akin to~\cite{Baez03b,FlVa00})}. For any~$\epsilon>0$, there exists a constant $C_\epsilon>0$ such that
\[
|d_n|<C_\epsilon k^{1/2+\epsilon}.
\]
\item[]{\bf RH (Riemann hypothesis)}. The Riemann 
zeta function~$\zeta(s)$ is free of zeros in the half-plane~$\Re(s)>\frac12$.
\end{itemize}
\end{theorem}
\begin{proof}
$(i)$~Assume {\bf RH}. Under RH, it is known that, given any~$\sigma_0>\frac12$ and any~$\epsilon>0$, 
one has $1/\zeta(s)=O(|t|^\epsilon)$ for $\Re(s)=\sigma_0$,
where $t=\Im(s)$ (see Equation~(14.2.6) of~\cite[p.~337]{Titchmarsh86}). 
Then, apply the N\"orlund integral representation (valid unconditionally for $c>1$):
\begin{equation}\label{norinv}
d_n=\frac{(-1)^{n-1}}{2\pi i}\int_{c-i\infty}^{c+i\infty}
\frac{1}{\zeta(s)} \frac{n!}{(s(s-1)\cdots(s-n)}\, ds,
\end{equation}
with $c=\sigma_0$.
A simple majorization concludes the proof  that $d_n=O(n^{\sigma_0})$.

$(ii)$~Assume {\bf FVBD}. First, a reorganization similar to the one leading to~\eqref{simpdel}
but based on the expansion of $1/\zeta(s)$ shows that
\[
d_n=\sum_{\ell=1}^\infty \mu(\ell)\left[\left(1-\frac{1}{\ell}\right)-1+\frac{n}{\ell}\right],
\]
with $\mu(\ell)$ the M\"obius function. The general term of the sum decreases like $n/\ell^2$,
which ensures absolute convergence. Next, introduce the function
\[
D(x)=\sum_{\ell\ge1} \mu(\ell)\left[e^{-x/\ell}-1+\frac{x}{\ell}\right],
\]
whose general term decreases like $x/\ell^2$. 

Fix any small~$\delta>0$ ($\delta=\frac1{10}$ is suitable) and define $\ell_0=\lfloor x^{1-\delta}\rfloor$.
The difference $d_n-D(n)$ satisfies
\begin{equation}\label{dede}
\begin{array}{lll}
 d_n-D(n) &=& \ds \sum_{\ell=1}^\infty \mu(\ell)\left[\left(1-\frac{1}{\ell}\right)^n
-e^{-n/\ell}\right]
% \\
%  &=& \ds \sum_{\ell=1}^\infty \mu(\ell)e^{-n/\ell}\left[e^{n/\ell+n\log(1-1/\ell)}
% -1\right]
\\
&=& \ds \left(\sum_{\ell<\ell_0}+\sum_{\ell\ge \ell_0}\right)
\mu(\ell)e^{-n/\ell}\left[e^{n/\ell+n\log(1-1/\ell)}
-1\right]
\\
&=&\ds \O\left(\ell_0 e^{-n/\ell_0}\right) +\sum_{\ell\ge\ell_0}\O\left(\frac{n}{\ell^2}\right)
 \quad = \quad\ds \O\left(n^\delta\right),
\end{array}
\end{equation}
by series reorganization, a split of the sum according to $\ell \gtreqless\ell_0$,
and trivial majorizations. 

Given~\eqref{dede}, the FVBD Hypothesis implies that $D(x)=\O(x^{1/2+\epsilon})$,
at least when $x$ is a positive \emph{integer}. To extend this estimate to real values of~$x$, 
it suffices to note that $D(x)$ is differentiable on~$\R_{>0}$ and
\[
D'(x)=\sum_{\ell=1}^\infty \frac{\mu(\ell)}{\ell}\left[e^{-x/\ell}-1\right],
\]
is proved to be $\O(1)$ by bounding techniques similar to~\eqref{dede}.
% 
% we consider
% the difference
% \begin{equation}
% \label{difd}
% D(x+h)-D(x)=\sum_{\ell=1}^\infty \mu(\ell) \left[e^{-(x+h)/\ell}-e^{-x/\ell}+\frac{h}{\ell}\right],
% \end{equation}
% for $h\in[0,1]$ and $x>0$. Since $\sum_{\ell\ge1}\mu(\ell)\ell^{-1}=1=0$, a property recognized to
% be equivalent to the prime number theorem, the term $h/\ell$ in~\eqref{difd} can be dropped,
% giving
% \[\begin{array}{lll}
% D(x+h)-D(x)&=& \ds \sum_{\ell=1}^\infty \mu(\ell) \left[e^{-(x+h)/\ell}-e^{-x/\ell}\right]\\
% &=& \ds \sum_{\ell=1}^\infty \mu(\ell) e^{-x/\ell} \left[-\frac{h}{\ell}+\O\left(\frac{1}{\ell^2}\right)\right]
% \quad=\quad \O(1).
% \end{array}\]
Thus, assuming the FVBD Hypothesis, the estimate
\[
D(x)=\O\left(x^{1/2+\epsilon}\right), \qquad x\to+\infty
\]
holds for \emph{real} values of~$x$.


Regarding the behaviour of~$D(x)$ at~$0$,
the general term of $D(x)$ is asymptotic to $x^2/(2\ell^2)$, so that $D(x)=\O(x^2)$,
as $x\to0^{+}$. This, combined with the estimate of $D(x)$ at infinity, implies that
the Mellin transform
\begin{equation}\label{mel1}
D^\star(s):=\int_0^\infty D(x) x^{s-1}\, dx,
\end{equation}
exists and is an analytic function of $s$ for all $s$ in the strip $-2<\Re(s)<-\frac12-\epsilon$.
On the other hand, the usual properties of Mellin transforms (see, e.g., the survey~\cite{FlGoDu95})
imply that 
\begin{equation}\label{mel2}
D^\star(s)=\left(\sum_{\ell=1}^\infty \mu(\ell)\ell^{s}\right)
\cdot \int_0^\infty e^{-x}x^{s-1}\, dx =\frac{\Gamma(s)}{\zeta(-s)},
\end{equation}
at least for $s$ such that $-2<\Re(s)<\-1$, which ensures that the expansion of $1/\zeta(-s)$
is absolutely convergent. The comparison of the analytic character of~\eqref{mel1} in
$-2<\Re(s)<-\frac12-\epsilon$ (implied by the FVBD Hypothesis) and of the explicit form of~\eqref{mel2}
shows that the Riemann Hypothesis is a consequence of the FVBD Hypothesis.
\end{proof}

Regarding the numerical properties of
the sequence~$d_n$, it appears that there are complicated oscillations, and these
 can to some extent be understood by a residue evaluation 
of~\eqref{norinv}, as we now explain. Assuming for notational convenience the simplicity 
of the nontrivial zeros of $\zeta(s)$, one has (unconditionally)
\begin{equation}\label{inv2}
d_n= \sum_{\rho}\!{}^{\hbox{$\star$}} \frac{1}{\zeta'(\rho)} 
\frac{\Gamma(n+1)\Gamma(-\rho)}{\Gamma(n+1-\rho)},
\end{equation}
where   the   summation   extends  to   all   nontrivial  zeros~$\rho$
of~$\zeta(s)$   with   $0<\Re(\rho)<1$,   while   the    starred   sum
($\sum{}\!{}^{\star}$) means  that  zeros should  be  suitably grouped
(see         the    discussion     in~\S9.8     of        Titchmarsh's
treatise~\cite[p.~219]{Titchmarsh86},   in  relation to  a  formula of
Ramanujan). A heuristic model of the sequence~$d_n$ then follows from the fact that,
for large~$n$, any individual term of the sum in~\eqref{inv2} 
corresponding to a zeta zero $\rho=\sigma+i\tau$
is  asymptotically
\[
\frac{\Gamma(-\rho)}{\zeta'(\rho)} n^{\sigma} e^{i\tau\log n}.\]
Such a term involves  a logarithmically oscillating  component,  a
slowly growing  component  $n^\sigma$ ($n^{\frac12}$ under  RH), as well as a
multiplier that  is likely   to  be extremely small, since    it involves
the quantity $\Gamma(-\rho)\asymp e^{-\pi|\tau|/2}$.  The  resulting   oscillations
then have the curious feature of being numerically detectable only for
rather large  values  of~$n$, and a  possible failure  of RH is
exponentially  offset---a       similar    fact   was     observed
in~\cite{FlVa00,Maslanka06}.  It is finally of  interest to note  that
such   phenomena   do  occur   in   nature,   specifically,   in   the
determination by~\cite{FlVa00} of  \emph{the    expected number  of  continued
fraction digits  that are  necessary to  sort $n$  real numbers
drawn uniformly at random from the unit interval}.


\bibliographystyle{plain}
\bibliography{algo}


\end{document}

